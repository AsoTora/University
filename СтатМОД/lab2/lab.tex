\documentclass[a4paper,hidelinks,14pt]{extarticle}

\input{sys/packages}        % Подключаемые пакеты
\input{sys/styles}   % Пользовательские стили

\begin{document}

\input{sys/names}	 % Переопределение именований
\begin{titlepage}
\thispagestyle{empty}
\setlength{\parindent}{0ex} % set paragraph indenting to zero

\begin{center}
  Министерство образования Республики Беларусь \\
  \vspace{0.5ex}
  Учреждение образования \\
  БЕЛОРУССКИЙ ГОСУДАРСТВЕННЫЙ УНИВЕРСИТЕТ \\
  ИНФОРМАТИКИ И РАДИОЭЛЕКТРОНИКИ \\
  \vspace{0.5ex}
  Факультет информационных технологий и управления \\
  \vspace{0.5ex}
  Кафедра ИТАС
\end{center}

\vspace{50mm}

\begin{center}
    Отчет по лабораторной работе №1 \\
    «Работа с \textit{HTML} и \textit{CSS}»

      \smallskip
      Вариант №18
\end{center}

\vspace{50mm}

\begin{minipage}{.4\linewidth}
    Выполнил ст. г. 820601

    \smallskip

    Проверил преп. каф. ИТАС
\end{minipage}
\hfill
\begin{minipage}{.4\linewidth}
    \begin{flushright}
        А. В. Пальчик

        \smallskip

         А. Г. Гончаревич
    \end{flushright}
\end{minipage}

\vfill
\begin{center}
  Минск \the\year{}
\end{center}

\setlength{\parindent}{1.25cm} % reset paragraph indenting
\end{titlepage}
	 % Титульный лист

%\input{text/aims}        % Цели работы
\section{ЦЕЛЬ РАБОТЫ}

\begin{itemize}
	\item Изучение многомерных распределений теории вероятностей и математической статистики.
	\item Исследование многомерных распределений теории вероятностей и математической статистики с помощью средств \textit{Matlab}.
\end{itemize}

\section{ЗАДАНИЕ}
\begin{itemize}
\item Вывести на экран монитора графики поверхностей и линии равных уровней плотностей вероятности двухмерных распределений (при ) из , указанных преподавателем, и исследовать их зависимость от параметров распределений.

\item Для нормального распределения в одно графическое окно вывести эллипс рассеяния и две функции регрессии. Исследовать зависимость формы и площади эллипса рассеяния от коэффициента корреляции при заданных дисперсиях компонент случайного вектора. Исследовать взаимное расположение функций регрессии и осей эллипса рассеяния (совпадают ли функции регрессии с осями эллипса?).
\end{itemize}

\newpage

%\input{text/process}     % Ход работы
\section{ХОД РАБОТЫ}

\subsection{Графики плотности распределения Фишера}
Построим график плотности распределения Фишера по формуле 1.2.8.6. Листинг кода программы приведён на Рисунке~\ref{lst:listing1}
\begin{lstlisting}[language=matlab,caption=Код программы \textit{Matlab}, label={lst:listing1}]
    clc
    x=[-10:0.01:10]
    y1=0;
    y2=0;
    y3=0;
    k1=1;
    k2=2;
    k3=100;
    m1=1;
    m2=1;
    m3=100;
    for i=1:length(x)
    if x(i)>0
    y1(i)=(gamma((k1+m1)/2)/(gamma(k1/2)*gamma(m1/2)))*(k1^(k1/2))*(m1^(m1/2))*(x(i)^((m1/2)-1))/((k1+m1*x(i))^((k1+m1)/2));
    y2(i)=(gamma((k2+m2)/2)/(gamma(k2/2)*gamma(m2/2)))*(k2^(k2/2))*(m2^(m2/2))*(x(i)^((m2/2)-1))/((k2+m2*x(i))^((k2+m2)/2));
    y3(i)=(gamma((k3+m3)/2)/(gamma(k3/2)*gamma(m3/2)))*(k3^(k3/2))*(m3^(m3/2))*(x(i)^((m3/2)-1))/((k3+m3*x(i))^((k3+m3)/2));
    else
    y1(i)=0;
    y2(i)=0;
    y3(i)=0;
    end
    end
    y4=fpdf(x,m1,k1);
    y5=fpdf(x,m2,k2);
    y6=fpdf(x,m3,k3);

    figure
    plot(x,y1,'r-*', x,y2,'b--', x,y3,'k-');
    grid on

    figure
    plot(x,y4,'r-*', x,y5,'b--', x,y6,'k-');
    grid on

\end{lstlisting}

Получен следующий график плотности распределения Фишера: Рисунок~\ref{fig:fig3}.

\begin{figure}[htbp]
    \centering
    \includegraphics[width=150mm,height=98mm,scale=0.7]{fig/raspr2.png}
    \caption{График плотности распределения Фишера, построенный по формуле}
    \label{fig:fig3}
\end{figure}


Построим график плотности распределения Фишера средствами \textit{MatLab}. Воспользуемся следующей функцией: $y=fpdf(x,m,k)$.

Листинг кода: \\
\begin{minipage}{\linewidth}
\begin{lstlisting}[language=matlab,caption=Код программы \textit{Matlab}, label={lst:listing2}]
    clc
    x=[-10:0.01:10]
    k1=1;
    k2=2;
    k3=100;
    m1=1;
    m2=1;
    m3=100;

    Y1 = fpdf(x,m1,k1)
    Y2= fpdf(x,m2,k2)
    Y3 = fpdf(x,m3,k3)

    figure
    plot(x,y1,'r-*', x,y2,'b--', x,y3,'k-');
    grid on
\end{lstlisting}
\end{minipage}

Получен следующий график плотности распределения Фишера: Рисунок~\ref{fig:fig4}.

\begin{figure}[!htbp]
    \centering
    \includegraphics[width=150mm,height=98mm]{fig/raspr3.png}
    \caption{График плотности распределения Фишера, построенный средствами \textit{Matlab}}
    \label{fig:fig4}
\end{figure}


\subsection{Графики функций распределения Фишера}

Построим графики функций распределения распределения Фишера средствами \textit{Matlab}.

\begin{minipage}{\linewidth}
    \begin{lstlisting}[language=matlab,caption=Код программы \textit{Matlab}, label={lst:listing3}]
clc
x=[-10:0.01:10]
y1=0;
y2=0;
y3=0;
k1=1;
k2=2;
k3=100;
m1=1;
m2=1;
m3=100;
y1 = fcdf(x, m1, k1);
y2 = fcdf(x, m2, k2);
y3 = fcdf(x, m3, k3);

plot(x,y1,'r-o',x,y2,'b--*', x, y3, 'k-')
grid on
\end{lstlisting}
\end{minipage}

Получен следующий график: Рисунок~\ref{fig:fig5}
\begin{figure}[!htbp]
    \centering
    \includegraphics[width=150mm,height=98mm]{fig/raspr_plotn.png}
    \caption{График плотности распределения Фишера, построенный средствами \textit{Matlab}}
    \label{fig:fig5}
\end{figure}


%\input{text/conclusion}  % Заключение
\section*{ЗАКЛЮЧЕНИЕ}

В ходе выполнения лабораторной работы ознакомились с системой  программирования \textit{Matlab}, приобрели навыки работы в ней. Ознакомились с языком программирования системы \textit{Matlab}.


\newpage


%\renewcommand{\thefigure}{\Asbuk{section}.\arabic{figure}}
\renewcommand{\thetable}{\Asbuk{section}.\arabic{table}}
\renewcommand{\thelstlisting}{\Asbuk{section}.\arabic{lstlisting}}

\chead{\vspace{1ex}Продолжение приложения А}
\pagestyle{fancy}
\thispagestyle{plain}

\section*{ПРИЛОЖЕНИЕ A\\(справочное)\\Листинг Кода Страниц}

\setcounter{section}{1}
\setcounter{figure}{0}
\setcounter{table}{0}
\setcounter{lstlisting}{0}

В редких случаях бывает удобно выделять объемные рисунки и таблицы, а также листиниги в приложения:

\begin{lstlisting}[language=HTML,caption=Исходный код страницы Index]
    <!DOCTYPE html>
    <html lang="en">
    <head>
    <meta charset="utf-8">
    <title>Weather.org</title>
    <link rel="stylesheet" href="style.css">
    </head>
    <body>
    <div class="all">
    <header class="header">
    <ul>
    <li style="margin-left: 20px;"><a id="mainlabel" href="index.html">Weather.org</a></li>
    <li class="dropdown">
    <div class="dropdown">
    <button class="dropbtn">World</button>
    <div class="dropdown-content">
    <a href="us.html">US</a>
    <a href="asia.html">Asia</a>
    <a href="africa.html">Africa</a>
    </div>
    </div>
    </li>
    <li><a href="history.html">Weather history</a></li>
    <li><a href="climateChange.html"></a></li>
    <li id="rightli"><a id="righta" href="contact.html">Contact</a></li>
    </ul>
    </header>
    <main>
    <div class="content">
    <center>
    <h1>Europe weather</h1>
    <p>Choose a country</p>
    <div style="margin-bottom: 30px;">
    <img src="images/europe.gif" alt="europe" usemap=#navigation>
    <map name=navigation>
    <area shape=circle coords="354,164,13"
    href=belarus.html alt="Belarus">
    <area shape=circle coords="300,176,20"
    href=poland.html alt="Poland">
    <area shape=circle coords="245,180,10"
    href=https://www.accuweather.com/en/browse-locations/eur/de alt="Germany">
    <area shape=rect coords="333,187,394,200"
    href=https://www.accuweather.com/en/browse-locations/eur/ua" alt="Ukraine">
    <area shape=circle coords="173,172,18"
    href=https://www.accuweather.com/en/browse-locations/eur/gb alt="Great Britan">
    <area shape=circle coords="197,221,22"
    href=https://www.accuweather.com/en/browse-locations/eur/fr alt="France">
    </map>
    </div>
    <strong><span style="color: #ff9900; text-align: left;">Latest conditions, forecasts for major European cities, resorts</span></strong>
    <ul class="listlink">
    <li>
    <div>
    <a  href="https://www.accuweather.com/en/browse-locations/eur/al">Albania</a>
    </div>
    </li>
    <li>
    <div>
    <a href="https://www.accuweather.com/en/browse-locations/eur/at" >Austria</a>
    </div>
    </li>
    <li>
    <div>
    <a href="https://www.accuweather.com/en/browse-locations/eur/be">Belgium</a>
    </div>
    </li>
    <li>
    <div>
    <a href="https://www.accuweather.com/en/browse-locations/eur/bg">Bulgaria</a>
    </div>
    </li>
    <li>
    <div>
    <a href="https://www.accuweather.com/en/browse-locations/eur/by">Belarus</a>
    </div>
    </li>
    <li>
    <div>
    <a href="https://www.accuweather.com/en/browse-locations/eur/ch">Switzerland</a>
    </div>
    </li>
    <li>
    <div>
    <a href="https://www.accuweather.com/en/browse-locations/eur/cz">Czechia</a>
    </div>
    </li>
    <li>
    <div>
    <a href="https://www.accuweather.com/en/browse-locations/eur/de">Germany</a>
    </div>
    </li>
    <li>
    <div>
    <a href="https://www.accuweather.com/en/browse-locations/eur/dk">Denmark</a>
    </div>
    </li>
    <li>
    <div>
    <a href="https://www.accuweather.com/en/browse-locations/eur/ee">Estonia</a>
    </div>
    </li>
    <li>
    <div>
    <a href="https://www.accuweather.com/en/browse-locations/eur/es">Spain</a>
    </div>
    </li>
    <li>
    <div>
    <a href="https://www.accuweather.com/en/browse-locations/eur/fi">Finland</a>
    </div>
    </li>
    <li>
    <div>
    <a href="https://www.accuweather.com/en/browse-locations/eur/fr">France</a>
    </div>
    </li>
    <li>
    <div>
    <a href="https://www.accuweather.com/en/browse-locations/eur/gb">United Kingdom</a>
    </div>
    </li>
    <li>
    <div>
    <a href="https://www.accuweather.com/en/browse-locations/eur/gr">Greece</a>
    </div>
    </li>
    <li>
    <div>
    <a href="https://www.accuweather.com/en/browse-locations/eur/hr">Croatia</a>
    </div>
    </li>
    <li>
    <div>
    <a href="https://www.accuweather.com/en/browse-locations/eur/hu">Hungary</a>
    </div>
    </li>
    <li>
    <div>
    <a href="https://www.accuweather.com/en/browse-locations/eur/ie">Ireland</a>
    </div>
    </li>
    <li>
    <div>
    <a href="https://www.accuweather.com/en/browse-locations/eur/it">Italy</a>
    </div>
    </li>
    <li>
    <div>
    <a href="https://www.accuweather.com/en/browse-locations/eur/lt">Lithuania</a>
    </div>
    </li>
    <li>
    <div>
    <a href="https://www.accuweather.com/en/browse-locations/eur/lv">Latvia</a>
    </div>
    </li>
    <li>
    <div>
    <a href="https://www.accuweather.com/en/browse-locations/eur/md">Moldova</a>
    </div>
    </li>
    <li>
    <div>
    <a href="https://www.accuweather.com/en/browse-locations/eur/nl">Netherlands</a>
    </div>
    </li>
    <li>
    <div>
    <a href="https://www.accuweather.com/en/browse-locations/eur/no">Norway</a>
    </div>
    </li>
    <li>
    <div>
    <a href="https://www.accuweather.com/en/browse-locations/eur/pl">Poland</a>
    </div>
    </li>
    <li>
    <div>
    <a href="https://www.accuweather.com/en/browse-locations/eur/pt">Portugal</a>
    </div>
    </li>
    <li>
    <div>
    <a href="https://www.accuweather.com/en/browse-locations/eur/ro">Romania</a>
    </div>
    </li>
    <li >
    <div>
    <a href="https://www.accuweather.com/en/browse-locations/eur/rs">Serbia</a>
    </div>
    </li>
    <li>
    <div>
    <a href="https://www.accuweather.com/en/browse-locations/eur/ru">Russia</a>
    </div>
    </li>
    <li>
    <div>
    <a href="https://www.accuweather.com/en/browse-locations/eur/se">Sweden</a>
    </div>
    </li>
    <li>
    <div>
    <a href="https://www.accuweather.com/en/browse-locations/eur/si">Slovenia</a>
    </div>
    </li>
    <li>
    <div>
    <a href="https://www.accuweather.com/en/browse-locations/eur/sk">Slovakia</a>
    </div>
    </li>
    <li>
    <div>
    <a href="https://www.accuweather.com/en/browse-locations/eur/tr">Turkey</a>
    </div>
    </li>
    <li>
    <div>
    <a href="https://www.accuweather.com/en/browse-locations/eur/ua">Ukraine</a>
    </div>
    </li>
    </ul>
    </div>
    </center>
    </main>
    </div>
    </body>
    </html>
\end{lstlisting}


\begin{lstlisting}[language=HTML,caption=Исходный код страницы Africa]
   <!DOCTYPE HTML PUBLIC "-//W3C//DTD HTML 4.01 Frameset//EN" "http://www.w3.org/TR/html4/frameset.dtd">
   <html lang="en">
   <head>
   <meta charset="utf-8">
   <title>Africa</title>
   <link rel="stylesheet" href="style.css">
   </head>
   <frameset rows="32%, *" noresize scrolling="no">
   <frame src="top.html"></frame>
   <frameset cols="840, *">
   <frame src="firstframe.html">
   </frame>
   <frame src="secondframe.html"></frame>
   </frameset>
   </frameset>
   </html>
\end{lstlisting}


\begin{lstlisting}[language=HTML,caption=Исходный код страницы Asia]
    <!DOCTYPE HTML PUBLIC "-//W3C//DTD HTML 4.01 Frameset//EN" "http://www.w3.org/TR/html4/frameset.dtd">
    <html lang="en">
    <head>
    <meta charset="utf-8">
    <title>Africa</title>
    <link rel="stylesheet" href="style.css">
    </head>
    <frameset rows="32%, *" noresize scrolling="no">
    <frame src="top.html"></frame>
    <frameset cols="840, *">
    <frame src="firstframe.html">
    </frame>
    <frame src="secondframe.html"></frame>
    </frameset>
    </frameset>
    </html><!DOCTYPE html>
    <html lang="en">
    <head>
    <meta charset="utf-8">
    <title>Asia</title>
    <link rel="stylesheet" href="style.css">
    </head>
    <body>
    <div class="all">
    <header class="header">
    <ul>
    <li style="margin-left: 20px;"><a id="mainlabel" href="index.html">Weather.org☔</a></li>
    <li class="dropdown">
    <div class="dropdown">
    <button class="dropbtn">World</button>
    <div class="dropdown-content">
    <a href="us.html">US</a>
    <a href="asia.html">Asia</a>
    <a href="africa.html">Africa</a>
    </div>
    </div>
    </li>
    <li><a href="history.html">Weather history</a></li>
    <li><a href="climateChange.html">Climate change</a></li>
    <li id="rightli"><a id="righta" href="contact.html">Contact</a></li>
    </ul>
    </header>
    <main>
    <div class="content">
    <center>
    <h1>Asia weather</h1>
    <img src="images/asia1.gif" style="margin-bottom:20px; width: 100%">
    <img src="images/asia2.gif" style="margin-bottom:20px; width: 100%;">
    <img src="images/asia.jpg" style="margin-bottom:20px;width: 100%;">
    </center>
    </div>
    </main>
    </div>
    </body>
    </html>
\end{lstlisting}


\begin{lstlisting}[language=HTML,caption=Исходный код страницы Belarus]
    <!DOCTYPE html>
    <html lang="en">
    <head>
    <meta charset="utf-8">
    <title>Belarus</title>
    <link rel="stylesheet" href="style.css">
    </head>
    <body>
    <div class="all">
    <header class="header">
    <ul>
    <li style="margin-left: 20px;"><a id="mainlabel" href="index.html">Weather.org☔</a></li>
    <li class="dropdown">
    <div class="dropdown">
    <button class="dropbtn">World</button>
    <div class="dropdown-content">
    <a href="us.html">US</a>
    <a href="asia.html">Asia</a>
    <a href="africa.html">Africa</a>
    </div>
    </div>
    </li>
    <li><a href="history.html">Weather history</a></li>
    <li><a href="climateChange.html">Climate change</a></li>
    <li id="rightli"><a id="righta" href="contact.html">Contact</a></li>
    </ul>
    </header>
    <main>
    <div class="content">
    <h1>Current weather, Belarus</h1>
    <table class="weathertable">
    <tr>
    <td>
    <div class="card">
    <span class="date">26.09</span>
    <div class="icon">
    <img src="images/cloudy.png" style="width: 150px;margin-top: 40px;">
    </div>
    <span class="temperature">+13</span>
    <div class="city">Minsk</div>
    </div>
    </td>
    <td>
    <div class="card">
    <span class="date">26.09</span>
    <div class="icon">
    <img src="images/sun.png" style="width: 150px;margin-top: 40px;">
    </div>
    <span class="temperature">+15</span>
    <div class="city">Brest</div>
    </div>
    </td>
    <td>
    <div class="card">
    <span class="date">26.09</span>
    <div class="icon">
    <img src="images/sun.png" style="width: 150px;margin-top: 40px;">
    </div>
    <span class="temperature">+15</span>
    <div class="city">Grodno</div>
    </div>
    </td>
    </tr>
    <tr>
    <td>
    <div class="card">
    <span class="date">26.09</span>
    <div class="icon">
    <img src="images/cloudy.png" style="width: 150px;margin-top: 40px;">
    </div>
    <span class="temperature">+12</span>
    <div class="city">Gomel</div>
    </div>
    </td>
    <td>
    <div class="card">
    <span class="date">26.09</span>
    <div class="icon">
    <img src="images/cloudy.png" style="width: 150px;margin-top: 40px;">
    </div>
    <span class="temperature">+11</span>
    <div class="city">Vitebsk</div>
    </div>
    </td>
    <td>
    <div class="card">
    <span class="date">26.09</span>
    <div class="icon">
    <img src="images/cloudy.png" style="width: 150px;margin-top: 40px;">
    </div>
    <span class="temperature">+11</span>
    <div class="city">Mogilev</div>
    </div>
    </td>
    </tr>
    </table>
    </div>
    </main>
    </div>
    </body>
    </html>
\end{lstlisting}



\begin{lstlisting}[language=HTML,caption=Исходный код страницы Poland]
   <!DOCTYPE html>
   <html lang="en">
   <head>
   <meta charset="utf-8">
   <title>Poland</title>
   <link rel="stylesheet" href="style.css">
   </head>
   <body>
   <div class="all">
   <header class="header">
   <ul>
   <li style="margin-left: 20px;"><a id="mainlabel" href="index.html">Weather.org☔</a></li>
   <li class="dropdown">
   <div class="dropdown">
   <button class="dropbtn">World</button>
   <div class="dropdown-content">
   <a href="us.html">US</a>
   <a href="asia.html">Asia</a>
   <a href="africa.html">Africa</a>
   </div>
   </div>
   </li>
   <li><a href="history.html">Weather history</a></li>
   <li><a href="climateChange.html">Climate change</a></li>
   <li id="rightli"><a id="righta" href="contact.html">Contact</a></li>
   </ul>
   </header>
   <main>
   <div class="content">
   <h1>Current weather, Poland</h1>
   <table class="weathertable">
   <tr>
   <td>
   <div class="card">
   <span class="date">26.09</span>
   <div class="icon">
   <img src="images/sun.png" style="width: 150px;margin-top: 40px;">
   </div>
   <span class="temperature">+19</span>
   <div class="city">Krakow</div>
   </div>
   </td>
   <td>
   <div class="card">
   <span class="date">26.09</span>
   <div class="icon">
   <img src="images/sun.png" style="width: 150px;margin-top: 40px;">
   </div>
   <span class="temperature">+16</span>
   <div class="city">Bialystok</div>
   </div>
   </td>
   <td>
   <div class="card">
   <span class="date">26.09</span>
   <div class="icon">
   <img src="images/cloudy.png" style="width: 150px;margin-top: 40px;">
   </div>
   <span class="temperature">+18</span>
   <div class="city">Warsaw</div>
   </div>
   </td>
   </tr>
   <tr>
   <td>
   <div class="card">
   <span class="date">26.09</span>
   <div class="icon">
   <img src="images/cloud.png" style="width: 150px;margin-top: 40px;">
   </div>
   <span class="temperature">+17</span>
   <div class="city">Zakopane</div>
   </div>
   </td>
   <td>
   <div class="card">
   <span class="date">26.09</span>
   <div class="icon">
   <img src="images/sun.png" style="width: 150px;margin-top: 40px;">
   </div>
   <span class="temperature">+14</span>
   <div class="city">Gdansk</div>
   </div>
   </td>
   <td>
   <div class="card">
   <span class="date">26.09</span>
   <div class="icon">
   <img src="images/sun.png" style="width: 150px;margin-top: 40px;">
   </div>
   <span class="temperature">+16</span>
   <div class="city">Lublin</div>
   </div>
   </td>
   </tr>
   <tr>
   <td>
   <div class="card">
   <span class="date">26.09</span>
   <div class="icon">
   <img src="images/sun.png" style="width: 150px;margin-top: 40px;">
   </div>
   <span class="temperature">+17</span>
   <div class="city">Katowice</div>
   </div>
   </td>
   <td>
   <div class="card">
   <span class="date">26.09</span>
   <div class="icon">
   <img src="images/sun.png" style="width: 150px;margin-top: 40px;">
   </div>
   <span class="temperature">+17</span>
   <div class="city">Lodz</div>
   </div>
   </td>
   <td>
   <div class="card">
   <span class="date">26.09</span>
   <div class="icon">
   <img src="images/sun.png" style="width: 150px;margin-top: 40px;">
   </div>
   <span class="temperature">+20</span>
   <div class="city">Poznan</div>
   </div>
   </td>
   </tr>
   </table>
   </div>
   </main>
   </div>
   </body>
   </html>
\end{lstlisting}



\begin{lstlisting}[language=HTML,caption=Исходный код страницы Climate Change]

    <!DOCTYPE html>
    <html lang="en">
    <head>
    <meta charset="utf-8">
    <title>History</title>
    <link rel="stylesheet" href="style.css">
    </head>
    <body>
    <div class="all">
    <header class="header">
    <ul>
    <li style="margin-left: 20px;"><a id="mainlabel" href="index.html">Weather.org</a></li>
    <li class="dropdown">
    <div class="dropdown">
    <button class="dropbtn">World</button>
    <div class="dropdown-content">
    <a href="us.html">US</a>
    <a href="asia.html">Asia</a>
    <a href="africa.html">Africa</a>
    </div>
    </div>
    </li>
    <li><a href="history.html">Weather history</a></li>
    <li><a href="climateChange.html">Climate change</a></li>
    <li id="rightli"><a id="righta" href="contact.html">Contact</a></li>
    </ul>
    </header>
    <main>
    <div class="content">
    <center><h1>Climate change</h1></center>
    <p><b>Climate change</b> includes both <b>global warming</b> driven by human-induced emissions of greenhouse gases and the resulting large-scale shifts in weather patterns.
    Though there have been previous periods of climatic change, since the mid-20th century humans have had an unprecedented impact on Earth's climate system and caused change on a global scale.
    </p>
    <p>The largest driver of warming is the carbon dioxide CO2 and methane.
    Fossil fuel burning coal, oil, and natural gas for energy consumption is the main source of these emissions, with additional contributions from agriculture, deforestation, and the chemical reactions in certain manufacturing processes.
    The human cause of climate change is not disputed by any scientific body of national or international standing.
    Temperature rise is amplified by climate feedbacks, such as loss of sunlight-reflecting snow and ice cover, increased water vapoura greenhouse gas itself, and changes to land and ocean carbon sinks.
    </p>
    <p>On land, where temperatures have risen about twice as fast as the global average, deserts are expanding and heat waves and wildfires are becoming more common.
    Temperature rise is also amplified in the Arctic, where it has contributed to melting permafrost, glacial retreat and sea ice loss.
    Warmer temperatures are increasing rates of evaporation, causing more intense storms and weather extremes.
    Impacts on ecosystems include the relocation or extinction of many species as their environment changes, most immediately in coral reefs, mountains, and the Arctic.
    Climate change threatens people with food insecurity, water scarcity, flooding, infectious diseases, extreme heat, economic losses, and displacement.
    These human impacts have led the World Health Organization to call climate change the greatest threat to global health in the 21st century.
    Even if efforts to minimise future warming are successful, some effects will continue for centuries, including rising sea levels</a>, rising ocean temperatures, and ocean acidification.
    </p>
    <p>Many of these impacts are already felt at the current level of warming, which is about 1.2°C 2.2°F.
    The Intergovernmental Panel on Climate Change</a> IPCC has issued a series of reports that project significant increases in these impacts as warming continues to 1.5°C 2.7°F and beyond.
    Additional warming also increases the risk of triggering critical thresholds called tipping points.
    Responding to these impacts involves both mitigation and adaptation.
    Mitigation – limiting climate change – consists of reducing greenhouse gas emissions and removing them from the atmosphere.
    Methods to achieve this include the development and deployment of low-carbon energy sources</a> such as wind and solar, a phase-out of coal</a>, enhanced energy efficiency, and forest preservation.
    Adaptation consists of adjusting to actual or expected climate, such as through improved coastline protection, better disaster management, and the development of more resistant crops.
    Adaptation alone cannot avert the risk of "severe, widespread and irreversible" impacts.
    </p>
    <p>Under the 2015 Paris Agreement, nations collectively agreed to keep warming "well under 2.0°C 3.6°F" through mitigation efforts. However, with pledges made under the Agreement, global warming would still reach about 2.8°C 5.0° by the end of the century.
    Limiting warming to 1.5°C 2.7°F would require halving emissions by 2030 and achieving near-zero emissions by 2050.
    </p>
    </div>
    </main>
    </div>
    </body>
    </html>

\end{lstlisting}


\begin{lstlisting}[language=HTML,caption=Исходный код стилей US]
<!DOCTYPE html>
<html lang="en">
<head>
<meta charset="utf-8">
<title>Weather.org</title>
<link rel="stylesheet" href="style.css">
</head>
<body>
<div class="all">
<header class="header">
<ul>
<li style="margin-left: 20px;"><a id="mainlabel" href="index.html">Weather.org</a></li>
<li class="dropdown">
<div class="dropdown">
<button class="dropbtn">World</button>
<div class="dropdown-content">
<a href="us.html">US</a>
<a href="asia.html">Asia</a>
<a href="africa.html">Africa</a>
</div>
</div>
</li>
<li><a href="history.html">Weather history</a></li>
<li><a href="climateChange.html">Climate change</a></li>
<li id="rightli"><a id="righta" href="contact.html">Contact</a></li>
</ul>
</header>
<main>
<div class="content">
<center>
<h1>US weather</h1>
<img src="images/us3.gif" style="margin-bottom:20px; width: 100%">
<img src="images/us1.png" style="margin-bottom:20px; width: 100%;">
<img src="images/us2.png" style="margin-bottom:20px;width: 100%;">
</center>
</div>
</main>
</div>
</body>
</html>
\end{lstlisting}



\begin{lstlisting}[language=HTML,caption=Исходный код страницы History]
    <!DOCTYPE html>
    <html lang="en">
    <head>
    <meta charset="utf-8">
    <title>History</title>
    <link rel="stylesheet" href="style.css">
    </head>
    <body>
    <div class="all">
    <header class="header">
    <ul>
    <li style="margin-left: 20px;"><a id="mainlabel" href="index.html">Weather.org</a></li>
    <li class="dropdown">
    <div class="dropdown">
    <button class="dropbtn">World</button>
    <div class="dropdown-content">
    <a href="us.html">US</a>
    <a href="asia.html">Asia</a>
    <a href="africa.html">Africa</a>
    </div>
    </div>
    </li>
    <li><a href="history.html">Weather history</a></li>
    <li><a href="climateChange.html">Climate change</a></li>
    <li id="rightli"><a id="righta" href="contact.html">Contact</a></li>
    </ul>
    </header>
    <main>
    <div class="content">
    <center><h1>Weather history</h1></center>
    <span>
    The list of periods and events in climate history includes some notable climate events known to paleoclimatology. Knowledge of precise climatic events decreases as the record goes further back in time. The timeline of glaciation covers ice ages specifically, which tend to have their own names for phases, often with different names used for different parts of the world. The names for earlier periods and events come from geology and paleontology. The marine isotope stages (MIS) are often used to express dating within the Quaternary.
    <br>
    <img src="images/climatechange.png" style="padding: 15px; width:100%; ">
    Before 1,000 Mya Faint young Sun paradox
    <br>
    2,400 Mya Great Oxygenation Event probably leads to Huronian glaciation perhaps covering the whole globe
    <br>
    650–600 Mya Later Neoproterozoic Snowball Earth or Marinoan glaciation, precursor to the Cambrian Explosion
    <br>
    517 Mya End-Botomian mass extinction; like the next two, little understood
    <br>
    502 Mya Dresbachian extinction event
    <br>
    485.4 Mya Cambrian–Ordovician extinction event
    <br>
    450–440 Mya Ordovician–Silurian extinction event, in two bursts, after cooling perhaps caused by tectonic plate movement
    <br>
    450 Mya Andean-Saharan glaciation
    <br>
    360-260 Mya Karoo Ice Age
    <br>
    305 Mya cooler climate causes Carboniferous Rainforest Collapse
    <br>
    251.902 Mya Permian-Triassic extinction event
    <br>
    199.6 Mya Triassic–Jurassic extinction event, causes as yet unclear
    <br>
    66 Mya, perhaps 30,000 years of volcanic activity form the Deccan Traps in India, Or a large meteor impact.
    <br>
    66 Mya Cretaceous–Paleogene boundary and Cretaceous–Paleogene extinction event, extinction of dinosaurs
    <br>
    55.8 Mya Paleocene-Eocene Thermal Maximum
    <br>
    53.7 Mya Eocene Thermal Maximum 2
    <br>
    49 Mya Azolla event may have ended a long warm period
    <br>
    5.3–2.6 Mya Pliocene climate became cooler and drier, and seasonal, similar to modern climates.
    <br>
    2.5 Mya to present Quaternary glaciation, with permanent ice on the polar regions, many named stages in different parts of the world
    </span>
    <img src="images/change2.png" style="padding: 15px; width:100%; ">
    </div>
    </main>
    </div>
    </body>
    </html>
\end{lstlisting}



\begin{lstlisting}[language=CSS,caption=Исходный код стилей CSS]
    body{
        background-color: rgba(229, 231, 231, 0.479);
        padding-left: 15%;
        padding-right: 15%;
        font-style: normal;
        font-family: "Trebuchet MS", sans-serif;
    }

    .all{
        background-color: rgb(255, 255, 255);
    }

    .header {
        font-size:large;
        font-weight: 800;
        background-color: rgb(247, 219, 62);
    }

    .header ul {
        list-style-type: none;
        margin: 0;
        padding: 0;
        overflow: hidden;
        vertical-align: middle;
    }

    .header li {
        float: left;
    }

    #mainlabel{
        font-weight: 900;
        font-style: oblique;
    }

    #rightli{
        float: right;
    }

    .header li a {
        border: 2px solid transparent;
        transition: all 0.1s linear;
        display: block;
        color: rgb(0, 0, 0);
        text-align: center;
        padding: 15px 15px;
        text-decoration: none;
    }

    .header li a:hover, .dropdown:hover .dropbtn {
        color: rgb(136, 136, 136);
    }

    .dropdown {
        float:inherit;
        overflow: hidden;
    }

    .dropdown .dropbtn {
        border: 2px solid transparent;
        font-size: inherit;
        font-weight: inherit;
        color: rgb(0, 0, 0);
        padding: 15px 15px;
        background-color: inherit;
        font-family: inherit;
    }

    .dropdown-content {
        display: none;
        position: absolute;
        background-color: rgb(247, 219, 62);
        width: 7%;
        z-index: 1;
    }

    .dropdown-content a:hover{
        color: rgb(136, 136, 136);
    }

    .dropdown-content a {
        float: none;
        color: black;
        padding: 15px 15px;
        text-decoration: none;
        display: block;
        text-align: left;
    }

    .dropdown:hover .dropdown-content {
        display: block;
    }

    .content {
        font-family: "Helvetica", sans-serif;
        margin-top: 25px;
        padding:20px;
        font-weight:200;
        font-size: medium;
        color: rgb(0, 0, 0);
        border: 1px solid #cfcfcf;
    }


    .listlink, .listlink a, .listlink a:visited{
        color: #3366ff;
        list-style-type: none;
    }

    .weathertable{
        width: 100%;
        border: 100px;
    }

    .weathertable td{
        padding: 10px;
    }

    .card{
        border: 1px solid #cfcfcf;
        display: inline-block;
        font-family: serif;
        text-align: center;
        font-weight: 700;
        width: 300px;
        height: 400px;
    }

    .card:hover{
        box-shadow: 0 0 7px black;
    }

    .date{
        display: flex;
        justify-content: center;
        align-items: center;
        height: 30px;
        background-color: #0097B9;
        color: white;
    }

    .icon{
        width: auto;
        height: 270;
        background-color: AliceBlue;
        margin: 0 auto 19px;
    }

    .temperature{
        font-size: 21px;
    }

    .temperature::after{
        content: "°";
    }

    .city {
        font-size: xx-large;
    }

    .myform {
        margin-left: 35%;
        width: 30%;
        margin-top: 6px;
        padding: 40px;
    }

    input[type=text], select, textarea {
        width: 100%;
        padding: 12px;
        border: 1px solid #ccc;
        border-radius: 4px;
        resize: vertical;
    }

    label {
        padding: 12px 12px 12px 0;
        display: inline-block;
    }

    input[type=submit] {
        background-color: #0fb4c0;
        color: white;
        padding: 12px 20px;
        border: none;
        border-radius: 4px;
        cursor: pointer;
        float: right;
    }

    input[type=submit]:hover {
        background-color: #0dd678;
    }
\end{lstlisting}




\newpage
    % Приложения
\end{document}
