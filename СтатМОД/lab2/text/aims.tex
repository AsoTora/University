\section{ЦЕЛЬ РАБОТЫ}

В цели работы обычно записывают копипасты из методы в виде ненумерованного списка,
так как в последующем тексте нет ссылок на эти самые цели.

Первая строка ненумерованного списка должна иметь абзацный отступ (от левого поля до тире); 
расстояние между элементами списка должно быть таким же, как между строками текста.

Всякий список всегда должен заканчиваться точкой. Если элементы списка --- 
предложения, то они должны быть записаны с большой буквы и заканчиваться точкой; если же это 
словосочетания, тогда они записываются с малой буквы и разделяются точкой с запятой, как, например, здесь: 

\begin{itemize} 
\item изучение принципов работы электронных вольтметров;
\item изучение методов измерения напряжений электронными вольтметрами;
\item изучение причин возникновения методических погрешностей, связанных с измерением напряжения переменного тока с помощью электронных вольтметров;
\item бла-бла-бла\dots, ненумерованный список должен заканчиваться точкой.
\end{itemize}

\newpage