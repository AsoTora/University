\section{ХОД РАБОТЫ}

Это самый важный и объёмный раздел работы. Уважающие себя студенты пишут его самостоятельно.
Хорошо написан ход работы --- считай, лаба наполовину сдана!

\subsection{Примеры оформления формул}

В соответствии со стандартом \textbf{требуется нумеровать все 
формулы}, которые расположены на отдельных строках; в крайнем случае допускается нумеровать
группы однотипных формул, при этом про необходимость ссылок на эти формулы в тексте ничего не сказано.

Еще одна тонкость: \textbf{ссылки на формулы должны быть заключены в скобки}, вот так: смотри выражение ~\eqref{eq:observational_error}.
\begin{equation}
  \label{eq:observational_error}
  \gamma_{i} = \dfrac{\Delta_{i}}{X_{N}}*100\;\%,
\end{equation}

\noindent где\hspace{1em}$ X_{N} $ --- нормируемое значение, которое согласно ГОСТ 8.401-80
следует выбирать равным пределу измерения;

$ Q $ --- действительное значение величины;

$ X_{i} $ --- показание прибора.
\begin{equation}
  \label{eq:BaseOpt}
  \begin{aligned}
    E &= 8(20x_{11} + 25x_{12}) + 7(28x_{21} + 18x_{22}) \rightarrow \max = \\
    &= 160x_{11} + 200x_{12} + 196x_{21} + 126x_{22} \rightarrow \max.
  \end{aligned}
\end{equation}
\begin{equation}
  \label{eq:BaseNSM}
    \begin{aligned}
      E = 160x_{11} + 200x_{12} &+ 196x_{21} + 126x_{22} \rightarrow \max, \\
      20x_{11} &+ 25x_{12} \ge 20, \\
      28x_{21} &+ 18x_{22} \ge 6, \\
      x_{11} &+ x_{21} \le 0{,}8, \\
      x_{12} &+ x_{22} \le 0{,}6, \\
      x_{ij} & \ge 0, i, j = 1, 2.
    \end{aligned}
\end{equation}
Дальше идет какой-то текст\dots

\newpage

\subsection{Пример оформления листинга}

По сути, листинг рассматривается в тексте документа как нечто среднее между рисунком и таблицей. Поправьте меня, если что не так: 
\begin{lstlisting}[language=c,caption=Исходный код какой-то программы на C]
#include <stdio.h>

int main() /* prints "Hello world!" */
{ 
  printf("Hello world!"); 
  return 0;
}
\end{lstlisting}
Так как листинг формально является рисунком, важно, чтобы расстояние
между этим <<рисунком>> и текстом составляло ровно одну пробельную строку.
